%
% Tutorial -- Introduction
%
% Copyright (C) 2005 Thierry Scordilis <thierry.scordilis@free.fr>
% Copyright (C) 2007 Stefan Jahn <stefan@lkcc.org>
%
% Permission is granted to copy, distribute and/or modify this document
% under the terms of the GNU Free Documentation License, Version 1.1
% or any later version published by the Free Software Foundation.
%


\section*{Important note and warning}

You should take into account the fact that this document is written on
the fly, so some mistakes are still possible, and the author is not
responsible for any damage due to the use of this document.

\bigskip

This document is intended to be a work book for RF and microwave
designers.  Our intention is not to provide an RF course, but some
touchy RF topics.  The goal is to insist on some design rules and work
flow for RF desings using CAD programs.  This work flow will be
handled through different chapters on quite different subjects.

\section*{Work book content}

In this workbook, we will pass through some regular tasks.  But there
is a progression on the explanations, and due to the fact that we have
to cover a huge amount of information, some key point will be shown
ony once, so it is recommanded to read the chapters in order.

\bigskip

This work book will include:

\begin{description}
\item[Work flow: ] the regular process of project design is shown,

\item[Understanding RF data sheets: ] a usual task, that could be
hell, could turn a project into a nightmare,

\item[BJT Modeling: ] after having chosen a device, we always need to
use in the CAD, and usually this device does not exits in the
CAD... how to create it and verify

\item[DC static: ] since all active devices have to be biased...

\item[PA Desgin: ] the active component is found, and a small
amplifier is designed without to many constraints

\item[LNA Design: ] a more constraint design using more rules,
stability, noise etc.

\item[oscillator design: ] a procedure that is typical from CAD
issues, handling non usual procedure,

\item[vco design: ] a normal evolution from a oscillator,

\item[detector: ] a design difficult to handle.

\item[more will come \ldots]

\end{description}
