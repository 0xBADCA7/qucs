%%%%%%%%%%%%%%%%%%%%%%%%%%%%%% LyX specific LaTeX commands.
%% Bold symbol macro for standard LaTeX users
\providecommand{\boldsymbol}[1]{\mbox{\boldmath $#1$}}

%% Because html converters don't know tabularnewline
\providecommand{\tabularnewline}{\\}

This chapter introduces the basic syntax of the function descriptions
and contains a categorical list of all available functions.

\tutsubsection{Functions Reference Format}

\char`\"{}Qucs\char`\"{} provides a rich set of functions, which can
be used to generate and display new datasets by function based evaluation
of simulation results. Beside a large number of mathematical standard
functions such as square root (sqrt), exponential function (exp),
absolute value (abs), functions especially useful for calculation
and transformation of electronic values are implemented. Examples
for the latter would be the conversion from Watts to dBm, the generation
of noise circles in an amplifier design, or the conversion from S-parameters
to Y-parameters.


\tutsubsubsection{Functions Reference Format}

In the subsequent two chapters, each function is described using the
following structure: 

\begin{description}
\item [<Function~Name>]~
\end{description}
Outlines briefly the functionality of the function.

\begin{description}
\item [Syntax]~
\end{description}
Defines the general syntax of this function.

\begin{description}
\item [Arguments]~
\end{description}
Name, type, definition range and whether the argument is optional,
are tabulated here. In case of an optional parameter the default value
is specified. {}``Type'' is a list defining the arguments allowed
and may contain the following symbols:
\medskip{}

\begin{tabular}{|c|c|}
\hline 
Symbol&
Description\tabularnewline
\hline
\hline 
$\mathbb{R}$&
Real number\tabularnewline
\hline 
$\mathbb{C}$&
Complex number\tabularnewline
\hline 
$\mathbb{R}^{n}$&
Vector consisting of \textit{n} real elements\tabularnewline
\hline 
$\mathbb{C}^{n}$&
Vector consisting of \textit{n} complex elements\tabularnewline
\hline
$\mathbb{R}^{m\times n}$&
Real matrix consisting of \textit{m} rows and \textit{n} columns\tabularnewline
\hline
$\mathbb{\mathbb{C}}^{m\times n}$&
Complex matrix consisting of \textit{m} rows and \textit{n} columns\tabularnewline
\hline
$\mathbb{R}^{m\times n\times p}$&
Vector of \textit{p} real $m\times n$ matrices\tabularnewline
\hline
$\mathbb{\mathbb{C}}^{m\times n\times p}$&
Vector of \textit{p} complex $m\times n$ matrices\tabularnewline
\hline
\end{tabular}

\medskip{}
{}``Definition range'' specifies the allowed range. Each range is
introduced by a bracket, either {}``$\left[\right.$'' or {}``$\left]\right.$'',
meaning that the following start value of the range is either included
or excluded. The start value is separated from the end value by a
comma. Then the end value follows, finished by a bracket again, either
{}``$\left[\right.$'' or {}``$\left]\right.$''. The first bracket
mentioned means {}``excluding the end value'', the second means
{}``including''.
\vspace{12pt}

If a range is given for a complex number, this specifies the real
or imaginary value of that number. If a range is given for a real
or complex vector or matrix, this specifies the real or imaginary
value of each element of that vector or matrix. The symbols mean {}``includes
listed value'' and {}``excludes listed value''.

\begin{description}
\item [Description]~
\end{description}
Gives a more detailed description on what the function does and what
it returns. In case some background knowledge is presented.

\begin{description}
\item [Examples]~
\end{description}
Shows an application of the function by one or several simple examples.

\begin{description}
\item [See~also]~
\end{description}
Shows links to related functions. A mouse click onto the desired link
leads to an immediate jump to that function.


\tutsubsection{Functions Listed by Category}

This compilation shows all {}``Qucs'' functions sorted by category.
Please click on the desired function to go to its detailed description.


\tutsubsubsection{Math Functions}


\subsubsection*{\nameref{sec:Vectors-and-Matrices}: \nameref{sub:Creation}}

\textcolor{blue}{}\begin{tabular}{>{\raggedleft}p{3cm}>{\centering}p{0.5cm}l}
\textcolor{blue}{\hyperlink{eye}{eye()}}&
...&
 \begin{NoHyper} \nameref{par:identity} \end{NoHyper}\tabularnewline
\textcolor{blue}{\hyperlink{linspace}{linspace()}}&
...&
 \begin{NoHyper} \nameref{par:linspace} \end{NoHyper}\tabularnewline
\textcolor{blue}{\hyperlink{logspace}{logspace()}}&
...&
 \begin{NoHyper} \nameref{par:logspace} \end{NoHyper}\tabularnewline
\end{tabular}


\subsubsection*{\nameref{sec:Vectors-and-Matrices}: \nameref{sub:Basic-Matrix-Functions}}

\textcolor{blue}{}\begin{tabular}{>{\raggedleft}p{3cm}>{\centering}p{0.5cm}l}
\textcolor{blue}{\hyperlink{adjoint}{adjoint()}}&
...&
 \begin{NoHyper} \nameref{par:Adjoint-matrix.} \end{NoHyper}\tabularnewline
\textcolor{blue}{\hyperlink{array}{array()}}&
...&
 \begin{NoHyper} \nameref{par:array} \end{NoHyper}\tabularnewline
\textcolor{blue}{\hyperlink{det}{det()}}&
...&
 \begin{NoHyper} \nameref{par:Determinant} \end{NoHyper}\tabularnewline
\textcolor{blue}{\hyperlink{inverse}{inverse()}}&
...&
 \begin{NoHyper} \nameref{par:Matrix-inverse} \end{NoHyper}\tabularnewline
\textcolor{blue}{\hyperlink{transpose}{transpose()}}&
...&
 \begin{NoHyper} \nameref{par:Matrix-transpose} \end{NoHyper}\tabularnewline
\end{tabular}


\subsubsection*{\nameref{sec:Elementary-Mathematical-Functions}: \nameref{sub:Basic-Real-and}}

\textcolor{blue}{}\begin{tabular}{>{\raggedleft}p{3cm}>{\centering}p{0.5cm}l}
\textcolor{blue}{\hyperlink{abs}{abs()}}&
...&
 \begin{NoHyper} \nameref{par:Absolute-value} \end{NoHyper}\tabularnewline
\textcolor{blue}{\hyperlink{angle}{angle()}}&
...&
 \begin{NoHyper} \nameref{par:angle} \end{NoHyper}\tabularnewline
\textcolor{blue}{\hyperlink{arg}{arg()}}&
...&
 \begin{NoHyper} \nameref{par:arg} \end{NoHyper}\tabularnewline
\textcolor{blue}{\hyperlink{conj}{conj()}}&
...&
 \begin{NoHyper} \nameref{par:Conjugate} \end{NoHyper}\tabularnewline
\textcolor{blue}{\hyperlink{deg2rad}{deg2rad()}}&
...&
 \begin{NoHyper} \nameref{par:deg2rad} \end{NoHyper}\tabularnewline
\textcolor{blue}{\hyperlink{imag}{imag()}}&
...&
 \begin{NoHyper} \nameref{par:Imag} \end{NoHyper}\tabularnewline
\textcolor{blue}{\hyperlink{mag}{mag()}}&
...&
 \begin{NoHyper} \nameref{par:Magnitude} \end{NoHyper}\tabularnewline
\textcolor{blue}{\hyperlink{norm}{norm()}}&
...&
 \begin{NoHyper} \nameref{par:norm} \end{NoHyper}\tabularnewline
\textcolor{blue}{\hyperlink{phase}{phase()}}&
...&
 \begin{NoHyper} \nameref{par:Phase} \end{NoHyper}\tabularnewline
\textcolor{blue}{\hyperlink{polar}{polar()}}&
...&
 \begin{NoHyper} \nameref{par:polar} \end{NoHyper}\tabularnewline
\textcolor{blue}{\hyperlink{rad2deg}{rad2deg()}}&
...&
 \begin{NoHyper} \nameref{par:rad2deg} \end{NoHyper}\tabularnewline
\textcolor{blue}{\hyperlink{real}{real()}}&
...&
 \begin{NoHyper} \nameref{par:Real} \end{NoHyper}\tabularnewline
\textcolor{blue}{\hyperlink{signum}{signum()}}&
...&
 \begin{NoHyper} \nameref{par:Signum} \end{NoHyper}\tabularnewline
\textcolor{blue}{\hyperlink{sign}{sign()}}&
...&
 \begin{NoHyper} \nameref{par:Sign} \end{NoHyper}\tabularnewline
\textcolor{blue}{\hyperlink{sqr}{sqr()}}&
...&
 \begin{NoHyper} \nameref{par:Square} \end{NoHyper}\tabularnewline
\textcolor{blue}{\hyperlink{sqrt}{sqrt()}}&
...&
 \begin{NoHyper} \nameref{par:Square-root} \end{NoHyper}\tabularnewline
\textcolor{blue}{\hyperlink{unwrap}{unwrap()}}&
...&
 \begin{NoHyper} \nameref{par:Unwrap} \end{NoHyper}\tabularnewline
\end{tabular}


\subsubsection*{\nameref{sec:Elementary-Mathematical-Functions}: \nameref{sub:Exponential-and-Logarithmic}}

\textcolor{blue}{}\begin{tabular}{>{\raggedleft}p{3cm}>{\centering}p{0.5cm}l}
\textcolor{blue}{\hyperlink{exp}{exp()}}&
...&
 \begin{NoHyper} \nameref{par:Exponential-function} \end{NoHyper}\tabularnewline
\textcolor{blue}{\hyperlink{log10}{log10()}}&
...&
 \begin{NoHyper} \nameref{par:Decimal-logarithm} \end{NoHyper}\tabularnewline
\textcolor{blue}{\hyperlink{log2}{log2()}}&
...&
 \begin{NoHyper} \nameref{par:Binary-logarithm} \end{NoHyper}\tabularnewline
\textcolor{blue}{\hyperlink{ln}{ln()}}&
...&
 \begin{NoHyper} \nameref{par:Natural-logarithm} \end{NoHyper}\tabularnewline
\end{tabular}


\subsubsection*{\nameref{sec:Elementary-Mathematical-Functions}: \nameref{sub:Trigonometry}}

\textcolor{blue}{}\begin{tabular}{>{\raggedleft}p{3cm}>{\centering}p{0.5cm}l}
\textcolor{blue}{\hyperlink{cos}{cos()}}&
...&
 \begin{NoHyper} \nameref{par:Cosine} \end{NoHyper}\tabularnewline
\textcolor{blue}{\hyperlink{cosec}{cosec()}}&
...&
 \begin{NoHyper} \nameref{par:Cosecant} \end{NoHyper}\tabularnewline
\textcolor{blue}{\hyperlink{cot}{cot()}}&
...&
 \begin{NoHyper} \nameref{par:Cotangent} \end{NoHyper}\tabularnewline
\textcolor{blue}{\hyperlink{sec}{sec()}}&
...&
 \begin{NoHyper} \nameref{par:Secant} \end{NoHyper}\tabularnewline
\textcolor{blue}{\hyperlink{sin}{sin()}}&
...&
 \begin{NoHyper} \nameref{par:Sine} \end{NoHyper}\tabularnewline
\textcolor{blue}{\hyperlink{tan}{tan()}}&
...&
 \begin{NoHyper} \nameref{par:Tangent} \end{NoHyper}\tabularnewline
\end{tabular}


\subsubsection*{\nameref{sec:Elementary-Mathematical-Functions}: \nameref{sub:Inverse-Trigonometric-Functions}}

\textcolor{blue}{}\begin{tabular}{>{\raggedleft}p{3cm}>{\centering}p{0.5cm}l}
\textcolor{blue}{\hyperlink{arccos}{arccos()}}&
...&
 \begin{NoHyper} \nameref{par:Arc-cosine} \end{NoHyper}\tabularnewline
\textcolor{blue}{\hyperlink{arccot}{arccot()}}&
...&
 \begin{NoHyper} \nameref{par:Arc-cotangent} \end{NoHyper}\tabularnewline
\textcolor{blue}{\hyperlink{arcsin}{arcsin()}}&
...&
 \begin{NoHyper} \nameref{par:Arc-sine} \end{NoHyper}\tabularnewline
\textcolor{blue}{\hyperlink{arctan}{arctan()}}&
...&
 \begin{NoHyper} \nameref{par:Arc-tangent} \end{NoHyper}\tabularnewline
\end{tabular}


\subsubsection*{\nameref{sec:Elementary-Mathematical-Functions}: \nameref{sub:Hyperbolic-Functions}}

\textcolor{blue}{}\begin{tabular}{>{\raggedleft}p{3cm}>{\centering}p{0.5cm}l}
\textcolor{blue}{\hyperlink{cosh}{cosh()}}&
...&
 \begin{NoHyper} \nameref{par:Hyperbolic-cosine} \end{NoHyper}\tabularnewline
\textcolor{blue}{\hyperlink{cosech}{cosech()}}&
...&
 \begin{NoHyper} \nameref{par:Hyperbolic-cosecant} \end{NoHyper}\tabularnewline
\textcolor{blue}{\hyperlink{coth}{coth()}}&
...&
 \begin{NoHyper} \nameref{par:Hyperbolic-cotangent} \end{NoHyper}\tabularnewline
\textcolor{blue}{\hyperlink{sech}{sech()}}&
...&
 \begin{NoHyper} \nameref{par:Hyperbolic-secant} \end{NoHyper}\tabularnewline
\textcolor{blue}{\hyperlink{sinh}{sinh()}}&
...&
 \begin{NoHyper} \nameref{par:Hyperbolic-sine} \end{NoHyper}\tabularnewline
\textcolor{blue}{\hyperlink{tanh}{tanh()}}&
...&
 \begin{NoHyper} \nameref{par:Hyperbolic-tangent} \end{NoHyper}\tabularnewline
\end{tabular}


\subsubsection*{\nameref{sec:Elementary-Mathematical-Functions}: \nameref{sub:Inverse-Hyperbolic-Functions}}

\textcolor{blue}{}\begin{tabular}{>{\raggedleft}p{3cm}>{\centering}p{0.5cm}l}
\textcolor{blue}{\hyperlink{arcosh}{arcosh()}}&
...&
 \begin{NoHyper} \nameref{par:Hyperbolic-area-cosine} \end{NoHyper}\tabularnewline
\textcolor{blue}{\hyperlink{arcoth}{arcoth()}}&
...&
 \begin{NoHyper} \nameref{par:Hyperbolic-area-cotangent} \end{NoHyper}\tabularnewline
\textcolor{blue}{\hyperlink{arsinh}{arsinh()}}&
...&
 \begin{NoHyper} \nameref{par:Hyperbolic-area-sine} \end{NoHyper}\tabularnewline
\textcolor{blue}{\hyperlink{artanh}{artanh()}}&
...&
 \begin{NoHyper} \nameref{par:Hyperbolic-area-tangent} \end{NoHyper}\tabularnewline
\end{tabular}


\subsubsection*{\nameref{sec:Elementary-Mathematical-Functions}: \nameref{sub:Rounding}}

\textcolor{blue}{}\begin{tabular}{>{\raggedleft}p{3cm}>{\centering}p{0.5cm}l}
\textcolor{blue}{\hyperlink{ceil}{ceil()}}&
...&
 \begin{NoHyper} \nameref{par:ceil} \end{NoHyper}\tabularnewline
\textcolor{blue}{\hyperlink{fix}{fix()}}&
...&
 \begin{NoHyper} \nameref{par:fix} \end{NoHyper}\tabularnewline
\textcolor{blue}{\hyperlink{floor}{floor()}}&
...&
 \begin{NoHyper} \nameref{par:floor} \end{NoHyper}\tabularnewline
\textcolor{blue}{\hyperlink{round}{round()}}&
...&
 \begin{NoHyper} \nameref{par:round} \end{NoHyper}\tabularnewline
\end{tabular}


\subsubsection*{\nameref{sec:Elementary-Mathematical-Functions}: \nameref{sub:Special-Mathematical-Functions}}

\textcolor{blue}{}\begin{tabular}{>{\raggedleft}p{3cm}>{\centering}p{0.5cm}l}
\textcolor{blue}{\hyperlink{besseli0}{besseli0()}}&
...&
 \begin{NoHyper} \nameref{par:Modified-Bessel-function} \end{NoHyper}\tabularnewline
\textcolor{blue}{\hyperlink{besselj}{besselj()}}&
...&
 \begin{NoHyper} \nameref{par:Bessel-function} \end{NoHyper}\tabularnewline
\textcolor{blue}{\hyperlink{bessely}{bessely()}}&
...&
 \begin{NoHyper} \nameref{par:Bessel2-function} \end{NoHyper}\tabularnewline
\textcolor{blue}{\hyperlink{erf}{erf()}}&
...&
 \begin{NoHyper} \nameref{par:Error-function} \end{NoHyper}\tabularnewline
\textcolor{blue}{\hyperlink{erfc}{erfc()}}&
...&
 \begin{NoHyper} \nameref{par:Complementary-error-function} \end{NoHyper}\tabularnewline
\textcolor{blue}{\hyperlink{erfinv}{erfinv()}}&
...&
 \begin{NoHyper} \nameref{par:Inverse-error-function} \end{NoHyper}\tabularnewline
\textcolor{blue}{\hyperlink{erfcinv}{erfcinv()}}&
...&
 \begin{NoHyper} \nameref{par:Inverse-complementary-error} \end{NoHyper}\tabularnewline
\textcolor{blue}{\hyperlink{sinc}{sinc()}}&
...&
 \begin{NoHyper} \nameref{par:Sinc-function} \end{NoHyper}\tabularnewline
\textcolor{blue}{\hyperlink{step}{step()}}&
...&
 \begin{NoHyper} \nameref{par:Step-function} \end{NoHyper}\tabularnewline
\end{tabular}


\subsubsection*{\nameref{sec:Data-Analysis}: \nameref{sub:Basic-Statistics}}

\textcolor{blue}{}\begin{tabular}{>{\raggedleft}p{3cm}>{\centering}p{0.5cm}l}
\textcolor{blue}{\hyperlink{avg}{avg()}}&
...&
 \begin{NoHyper} \nameref{par:Average} \end{NoHyper}\tabularnewline
\textcolor{blue}{\hyperlink{cumavg}{cumavg()}}&
...&
 \begin{NoHyper} \nameref{par:Cumulative-average} \end{NoHyper}\tabularnewline
\textcolor{blue}{\hyperlink{max}{max()}}&
...&
 \begin{NoHyper} \nameref{par:Maximum} \end{NoHyper}\tabularnewline
\textcolor{blue}{\hyperlink{min}{min()}}&
...&
 \begin{NoHyper} \nameref{par:Minimum} \end{NoHyper}\tabularnewline
\textcolor{blue}{\hyperlink{rms}{rms()}}&
...&
 \begin{NoHyper} \nameref{par:rms} \end{NoHyper}\tabularnewline
\textcolor{blue}{\hyperlink{runavg}{runavg()}}&
...&
 \begin{NoHyper} \nameref{par:Running-average} \end{NoHyper}\tabularnewline
\textcolor{blue}{\hyperlink{stddev}{stddev()}}&
...&
 \begin{NoHyper} \nameref{par:Standard-deviation} \end{NoHyper}\tabularnewline
\textcolor{blue}{\hyperlink{variance}{variance()}}&
...&
 \begin{NoHyper} \nameref{par:Variance} \end{NoHyper}\tabularnewline
\end{tabular}


\subsubsection*{\nameref{sec:Data-Analysis}: \nameref{sub:Basic-Operation}}

\textcolor{blue}{}\begin{tabular}{>{\raggedleft}b{3cm}>{\centering}b{0.5cm}>{\raggedright}b{12cm}}
\textcolor{blue}{\hyperlink{cumprod}{cumprod()}}&
...&
 \begin{NoHyper} \nameref{par:Cumulative-product} \end{NoHyper}\tabularnewline
\textcolor{blue}{\hyperlink{cumsum}{cumsum()}}&
...&
 \begin{NoHyper} \nameref{par:Cumulative-sum} \end{NoHyper}\tabularnewline
\textcolor{blue}{\hyperlink{interpolate}{interpolate()}}&
...&
 \begin{NoHyper} \nameref{par:spline-interpolation} \end{NoHyper}\tabularnewline
\textcolor{blue}{\hyperlink{prod}{prod()}}&
...&
 \begin{NoHyper} \nameref{par:Prod} \end{NoHyper}\tabularnewline
\textcolor{blue}{\hyperlink{sum}{sum()}}&
...&
 \begin{NoHyper} \nameref{par:Sum} \end{NoHyper}\tabularnewline
\textcolor{blue}{\hyperlink{xvalue}{xvalue()}}

\textcolor{blue}{~}&
...

\textcolor{blue}{~}&
 \begin{NoHyper} \nameref{par:xvalue} \end{NoHyper}\tabularnewline
\textcolor{blue}{\hyperlink{yvalue}{yvalue()}}

\textcolor{blue}{~}&
...

\textcolor{blue}{~}&
 \begin{NoHyper} \nameref{par:yvalue} \end{NoHyper}\tabularnewline
\end{tabular}


\subsubsection*{\nameref{sec:Data-Analysis}: \nameref{sub:Differentiation-and-Integration}}

\textcolor{blue}{}\begin{tabular}{>{\raggedleft}p{3cm}>{\centering}p{0.5cm}l}
\textcolor{blue}{\hyperlink{diff}{diff()}}&
...&
 \begin{NoHyper} \nameref{par:Differentiate} \end{NoHyper}\tabularnewline
\textcolor{blue}{\hyperlink{integrate}{integrate()}}&
...&
 \begin{NoHyper} \nameref{par:Integrate} \end{NoHyper}\tabularnewline
\end{tabular}


\subsubsection*{\nameref{sec:Data-Analysis}: \nameref{sub:Signal-Processing}}

\textcolor{blue}{}\begin{tabular}{>{\raggedleft}p{3cm}>{\centering}p{0.5cm}l}
\textcolor{blue}{\hyperlink{dft}{dft()}}&
...&
 \begin{NoHyper} \nameref{par:Discrete-Fourier-Transform} \end{NoHyper}\tabularnewline
\textcolor{blue}{\hyperlink{fft}{fft()}}&
...&
 \begin{NoHyper} \nameref{par:Fast-Fourier-Transform} \end{NoHyper}\tabularnewline
\textcolor{blue}{\hyperlink{idft}{idft()}}&
...&
 \begin{NoHyper} \nameref{par:Inverse-Discrete-Fourier} \end{NoHyper}\tabularnewline
\textcolor{blue}{\hyperlink{ifft}{ifft()}}&
...&
 \begin{NoHyper} \nameref{par:Inverse-Fast-Fourier} \end{NoHyper}\tabularnewline
\textcolor{blue}{\hyperlink{Time2Freq}{Time2Freq()}}&
...&
 \begin{NoHyper} \nameref{par:Interpreted-Discrete-Fourier-Transform} \end{NoHyper}\tabularnewline
\textcolor{blue}{\hyperlink{Freq2Time}{Freq2Time()}}&
...&
 \begin{NoHyper} \nameref{par:Interpreted-Inverse-Discrete-Fourier} \end{NoHyper}\tabularnewline
\textcolor{blue}{\hyperlink{kbd}{kbd()}}&
...&
 \begin{NoHyper} \nameref{par:Kaiser-Bessel-window} \end{NoHyper}\tabularnewline
\end{tabular}


\tutsubsubsection{Electronics Functions}


\subsubsection*{\nameref{sec:Unit-Conversion}}

\textcolor{blue}{}\begin{tabular}{>{\raggedleft}p{3cm}>{\centering}p{0.5cm}l}
\textcolor{blue}{\hyperlink{dB}{dB()}}&
...&
 \begin{NoHyper} \nameref{par:dB} \end{NoHyper}\tabularnewline
\textcolor{blue}{\hyperlink{dbm}{dbm()}}&
...&
 \begin{NoHyper} \nameref{par:dbm} \end{NoHyper}\tabularnewline
\textcolor{blue}{\hyperlink{dbm2w}{dbm2w()}}&
...&
 \begin{NoHyper} \nameref{par:dbm2w} \end{NoHyper}\tabularnewline
\textcolor{blue}{\hyperlink{w2dbm}{w2dbm()}}&
...&
 \begin{NoHyper} \nameref{par:w2dbm} \end{NoHyper}\tabularnewline
\end{tabular}


\subsubsection*{\nameref{sec:Reflection-Coefficients-and}}

\textcolor{blue}{}\begin{tabular}{>{\raggedleft}p{3cm}>{\centering}p{0.5cm}l}
\textcolor{blue}{\hyperlink{rtoswr}{rtoswr()}}&
...&
 \begin{NoHyper} \nameref{par:rtoswr} \end{NoHyper}\tabularnewline
\textcolor{blue}{\hyperlink{rtoy}{rtoy()}}&
...&
 \begin{NoHyper} \nameref{par:rtoy} \end{NoHyper}\tabularnewline
\textcolor{blue}{\hyperlink{rtoz}{rtoz()}}&
...&
 \begin{NoHyper} \nameref{par:rtoz} \end{NoHyper}\tabularnewline
\textcolor{blue}{\hyperlink{ytor}{ytor()}}&
...&
 \begin{NoHyper} \nameref{par:ytor} \end{NoHyper}\tabularnewline
\textcolor{blue}{\hyperlink{ztor}{ztor()}}&
...&
 \begin{NoHyper} \nameref{par:ztor} \end{NoHyper}\tabularnewline
\end{tabular}


\subsubsection*{\nameref{sec:N-Port-Matrix-Conversions}}

\textcolor{blue}{}\begin{tabular}{>{\raggedleft}b{3cm}>{\centering}b{0.5cm}>{\raggedright}b{12cm}}
\textcolor{blue}{\hyperlink{stos}{stos()}}

\textcolor{blue}{~}&
...

\textcolor{blue}{~}&
 \begin{NoHyper} \nameref{par:stos} \end{NoHyper}\tabularnewline
\textcolor{blue}{\hyperlink{stoy}{stoy()}}&
...&
 \begin{NoHyper} \nameref{par:stoy} \end{NoHyper}\tabularnewline
\textcolor{blue}{\hyperlink{stoz}{stoz()}}&
...&
 \begin{NoHyper} \nameref{par:stoz} \end{NoHyper}\tabularnewline
\textcolor{blue}{\hyperlink{twoport}{twoport()}}&
...&
 \begin{NoHyper} \nameref{par:twoport} \end{NoHyper}\tabularnewline
\textcolor{blue}{\hyperlink{ytos}{ytos()}}&
...&
 \begin{NoHyper} \nameref{par:ytos} \end{NoHyper}\tabularnewline
\textcolor{blue}{\hyperlink{ytoz}{ytoz()}}&
...&
 \begin{NoHyper} \nameref{par:ytoz} \end{NoHyper}\tabularnewline
\textcolor{blue}{\hyperlink{ztos}{ztos()}}&
...&
 \begin{NoHyper} \nameref{par:ztos} \end{NoHyper}\tabularnewline
\textcolor{blue}{\hyperlink{ztoy}{ztoy()}}&
...&
 \begin{NoHyper} \nameref{par:ztoy} \end{NoHyper}\tabularnewline
\end{tabular}


\subsubsection*{\nameref{sec:Amplifiers}}

\textcolor{blue}{}\begin{tabular}{>{\raggedleft}b{3cm}>{\centering}b{0.5cm}>{\raggedright}b{12cm}}
\textcolor{blue}{\hyperlink{GaCircle}{GaCircle()}}&
...&
 \begin{NoHyper} \nameref{par:GaCircle} \end{NoHyper}\tabularnewline
\textcolor{blue}{\hyperlink{GpCircle}{GpCircle()}}&
...&
 \begin{NoHyper} \nameref{par:GpCircle} \end{NoHyper}\tabularnewline
\textcolor{blue}{\hyperlink{Mu}{Mu()}}&
...&
 \begin{NoHyper} \nameref{par:Mu-stability-factor} \end{NoHyper}\tabularnewline
\textcolor{blue}{\hyperlink{Mu2}{Mu2()}}&
...&
 \begin{NoHyper} \nameref{par:Mu2-stability-factor} \end{NoHyper}\tabularnewline
\textcolor{blue}{\hyperlink{NoiseCircle}{NoiseCircle()}}&
...&
 \begin{NoHyper} \nameref{par:NoiseCircle} \end{NoHyper}\tabularnewline
\textcolor{blue}{\hyperlink{PlotVs}{PlotVs()}}

\textcolor{blue}{~}&
...

\textcolor{blue}{~}&
 \begin{NoHyper} \nameref{par:PlotVs} \end{NoHyper}\tabularnewline
\textcolor{blue}{\hyperlink{Rollet}{Rollet()}}&
...&
 \begin{NoHyper} \nameref{par:Rollet-stability-factor} \end{NoHyper}\tabularnewline
\textcolor{blue}{\hyperlink{StabCircleL}{StabCircleL()}}&
...&
 \begin{NoHyper} \nameref{par:StabCircleL} \end{NoHyper}\tabularnewline
\textcolor{blue}{\hyperlink{StabCircleS}{StabCircleS()}}&
...&
 \begin{NoHyper} \nameref{par:StabCircleS} \end{NoHyper}\tabularnewline
\end{tabular}
% \end{document}
