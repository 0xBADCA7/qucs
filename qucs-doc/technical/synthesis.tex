%
% This document contains the chapter about synthesizing circuits.
%
% Copyright (C) 2005, 2006 Stefan Jahn <stefan@lkcc.org>
% Copyright (C) 2005 Michael Margraf <Michael.Margraf@alumni.TU-Berlin.DE>
%
% Permission is granted to copy, distribute and/or modify this document
% under the terms of the GNU Free Documentation License, Version 1.1
% or any later version published by the Free Software Foundation.
%

\chapter{Synthesizing circuits}
\label{sec:synthesis}

\section{Attenuators}

Attenuators are used to damp a signal. Using pure ohmic resistors
the circuit can be realized for a very high bandwidth, i.e. from
DC to many GHz. The power attenuation $0 < L\le 1$ is defined as:
\begin{equation}
\label{eqn:loss}
L = \dfrac{P_{in}}{P_{out}}
  = \dfrac{V_{in}^2}{Z_{in}}\cdot\dfrac{Z_{out}}{V_{out}^2}
  = \left( \dfrac{V_{in}}{V_{out}}\right)^2 \cdot\dfrac{Z_{out}}{Z_{in}}
\end{equation}
where $P_{in}$ and $P_{out}$ are the input and output power and
$V_{in}$ and $V_{out}$ are the input and output voltages.

\begin{figure}[ht]
\begin{center}
\includegraphics[width=5cm]{picircuit}
\end{center}
\caption{$\pi$-topology of an attenuator}
\label{fig:pi_attenuator}
\end{figure}
\FloatBarrier

Fig. \ref{fig:pi_attenuator} shows an attenuator using the
$\pi$-topology. The conductances can be calculated as follows.
\begin{align}
Y_2 & = \dfrac{L - 1}{2\cdot \sqrt{L\cdot Z_{in}\cdot Z_{out}}} \\
Y_1 & = Y_2\cdot\left( \sqrt{\dfrac{Z_{out}}{Z_{in}}\cdot L} - 1 \right) \\
Y_3 & = Y_2\cdot\left( \sqrt{\dfrac{Z_{in}}{Z_{out}}\cdot L} - 1 \right)
\end{align}
where $Z_{in}$ and $Z_{out}$ are the input and output reference
impedances, respectively. The $\pi$-attenuator can be used for an
impedance ratio of:
\begin{equation}
\dfrac{1}{L} \le \dfrac{Z_{out}}{Z_{in}} \le L
\end{equation}

\begin{figure}[ht]
\begin{center}
\includegraphics[width=5cm]{tcircuit}
\end{center}
\caption{T-topology of an attenuator}
\label{fig:t_attenuator}
\end{figure}
\FloatBarrier

Fig. \ref{fig:t_attenuator} shows an attenuator using the
T-topology. The resistances can be calculated as follows.
\begin{align}
Z_2 & = \dfrac{2\cdot \sqrt{L\cdot Z_{in}\cdot Z_{out}}}{L - 1} \\
Z_1 & = Z_{in}\cdot A - Z_2 \\
Z_3 & = Z_{out}\cdot A - Z_2 \\
\textrm{with} \qquad A & = \dfrac{L + 1}{L - 1}
\end{align}
where $L$ is the attenuation ($0 < L\le 1$) according to
equation \ref{eqn:loss} and $Z_{in}$ and $Z_{out}$
are the input and output reference impedance, respectively.
The T-attenuator can be used for an impedance ratio of:
\begin{equation}
\dfrac{Z_{out}}{Z_{in}} \le \dfrac{(L+1)^2}{4\cdot L}
\end{equation}

\section{Filters}

One of the most common tasks in microwave technologies is to
extract a frequency band from others. Optimized filters exist
in order to easily create a filter with an appropriate characteristic.
The most popular ones are:

\addvspace{12pt}

\begin{tabular}{l|l}
Name & Property \\
\hline
Bessel filter (Thomson filter) & as constant group delay as possible \\
Butterworth filter (power-term filter) & as constant amplitude transfer function as possible \\
Chebychev filter type I & constant ripple in pass band \\
Chebychev filter type II & constant ripple in stop band \\
Cauer filter (elliptical filter) & constant ripple in pass and stop band \\
\end{tabular} 

\addvspace{12pt}

From top to bottom the following properties increase:
\begin{itemize}
\item ringing of step response
\item phase distortion
\item steepness of amplitude transfer function at the beginning of the pass band
\end{itemize}

\addvspace{12pt}

The order $n$ of a filter denotes the number of poles of its (voltage)
transfer function. It is:
\begin{equation}
\text{slope of asymptote} = \pm\, n\cdot 20 \text{dB/decade}
\end{equation}
Note that this equation holds for all filter characteristics, but
there are big differences concerning the attenuation near the pass
band.


\subsection{LC ladder filters}

The best possibility to realize a filters in VHF and UHF bands are
LC ladder filters. The usual way to synthesize them is to first
calculate a low-pass (LP) filter and afterwards transform it into a
high-pass (HP), band-pass (BP) or band-stop (BS) filter. To do so,
each component must be transformed into another.

\addvspace{12pt}

In a low-pass filter, there are  parallel capacitors $C_{LP}$ and
series inductors $L_{LP}$ in alternating order. The other filter
classes can be derived from it:

\addvspace{12pt}

In a high-pass filter:
\begin{align}
C_{LP} \quad \rightarrow \quad & L_{HP} = \dfrac{1}{\omega_B^2\cdot C_{LP}} \\
L_{LP} \quad \rightarrow \quad & C_{HP} = \dfrac{1}{\omega_B^2\cdot L_{LP}}
\end{align}

\addvspace{12pt}

In a band-pass filter:
\begin{align}
C_{LP} \quad \rightarrow \quad & \text{parallel resonance circuit with} \\
                               & C_{BP} = \dfrac{C_{LP}}{\Delta\Omega} \\
                               & L_{BP} = \dfrac{\Delta\Omega}{\omega_1\cdot \omega_2\cdot C_{LP}} \\
L_{LP} \quad \rightarrow \quad & \text{series resonance circuit with} \\
                               & C_{BP} = \dfrac{\Delta\Omega}{\omega_1\cdot \omega_2\cdot L_{LP}} \\
                               & L_{BP} = \dfrac{L_{LP}}{\Delta\Omega}
\end{align}

\addvspace{12pt}

In a band-stop filter:
\begin{align}
C_{LP} \quad \rightarrow \quad & \text{series resonance circuit with} \\
       & C_{BP} = \dfrac{C_{LP}}{2\cdot\left| \dfrac{\omega_2}{\omega_1} - \dfrac{\omega_1}{\omega_2} \right| } \\
       & L_{BP} = \dfrac{1}{\omega^2\cdot \Delta\Omega\cdot C_{LP}} \\
L_{LP} \quad \rightarrow \quad & \text{parallel resonance circuit with} \\
       & C_{BP} = \dfrac{1}{\omega^2\cdot \Delta\Omega\cdot L_{LP}} \\
       & L_{BP} = \dfrac{L_{LP}}{2\cdot\left| \dfrac{\omega_2}{\omega_1} - \dfrac{\omega_1}{\omega_2} \right| }
\end{align}

\addvspace{12pt}

Where
\begin{align}
\omega_1 \quad\rightarrow\quad & \text{lower corner frequency of frequency band} \\
\omega_2 \quad\rightarrow\quad & \text{upper corner frequency of frequency band} \\
\omega   \quad\rightarrow\quad & \text{center frequency of frequency band} \quad \omega = 0.5\cdot (\omega_1 + \omega_2) \\
\Delta\Omega \quad\rightarrow\quad & \Delta\Omega = \dfrac{|\omega_2 - \omega_1|}{\omega}
\end{align}

\subsubsection{Butterworth}

The $k$-th element of an $n$ order Butterworth low-pass ladder filter is:
\begin{alignat}{3}
 & \text{capacitance:} \qquad & C_k = & \dfrac{X_k}{Z_0} \\
 & \text{inductance:}  \qquad & L_k = & X_k \cdot Z_0 \\
 & \text{with}         \qquad & X_k = & \dfrac{2}{\omega_B} \cdot \sin \dfrac{(2\cdot k + 1)\cdot\pi}{2\cdot n}
\end{alignat}

\subsubsection{Chebyshev I}

The equations for a Chebyshev type I filter are defined recursivly.
With $R_{dB}$ being the passband ripple in decibel, the $k$-th
element of an $n$ order low-pass ladder filter is:
\begin{alignat}{3}
 & \text{capacitance:} \qquad & C_k & = \dfrac{X_k}{Z_0} \\
 & \text{inductance:}  \qquad & L_k & = X_k \cdot Z_0 \\
 & \text{with}         \qquad & X_k & = \dfrac{2}{\omega_B}\cdot g_k \\
 & & r   & = \sinh\left( \frac{1}{n}\cdot\text{arsinh}\dfrac{1}{\sqrt{10^{R_{dB}/10} - 1}} \right) \\
 & & a_k & = \sin \dfrac{(2\cdot k + 1)\cdot\pi}{2\cdot n} \\
 & & g_k & =
\begin{cases}
\begin{array}{ll}
  \dfrac{a_k}{r} & \textrm{ for } \quad k=0\\
  \dfrac{a_{k-1}\cdot a_k}{g_{k-1}\cdot \left( r^2 + \sin^2\dfrac{k\cdot\pi}{n} \right)} & \textrm{ for } \quad k\ge 1
\end{array}
\end{cases} \\
 & & X_k & = \dfrac{2}{\omega_B}\cdot g_k
\end{alignat}
