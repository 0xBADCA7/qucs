%
% This document contains the chapter about harmonic balance noise analysis.
%
% Copyright (C) 2005 Stefan Jahn <stefan@lkcc.org>
% Copyright (C) 2005 Michael Margraf <Michael.Margraf@alumni.TU-Berlin.DE>
%
% Permission is granted to copy, distribute and/or modify this document
% under the terms of the GNU Free Documentation License, Version 1.1
% or any later version published by the Free Software Foundation.
%

\chapter{Harmonic Balance Noise Analysis}
\label{sec:hb_noise}

Once a harmonic balance simulation is solved a cyclostationary
noise analysis can
be performed. This results in the sideband noise of each harmonic
(including DC, i.e. base band noise). The method described here
is based on the principle of small-signal noise. That is, the
noise power is assumed small enough (compared to the signal power
and its harmonics) to neglect the mixing of noise with noise.
This procedure is the standard concept in CAE and allows for a
quite simple and time-saving algorithm: Use the Jacobian to
calculate a conversion matrix and then apply the noise correlation
matrix to it.

\addvspace{12pt}

Figure \ref{fig:hbn_concept} shows the equivalent circuit for
starting the HB noise analysis. At every connection between
linear and non-linear subcircuit, there are two noise current
sources: one stemming from the linear subcircuit and one
stemming from the non-linear subcircuit.

\begin{figure}[ht]
\begin{center}
\includegraphics[width=9cm]{hbn_concept}
\end{center}
\caption{principle of harmonic balance noise model}
\label{fig:hbn_concept}
\end{figure}
\FloatBarrier


\section{The Linear Subcircuit}

The noise stemming from the linear subcircuit is calculated
in two steps:

\begin{enumerate}
\item An AC noise analysis (see section \ref{sec:acnoise_algo}) is
   performed for the interconnecting nodes of linear and non-linear
   subcircuit. This results in the noise-voltage correlation matrix
   $(\boldsymbol{\underline{C}}_{Z,lin})_{N\times N}$.
\item The matrix $(\boldsymbol{\underline{C}}_{Z,lin})_{N\times N}$
   is converted into a noise-current correlation matrix (see section
   \ref{sec:noiseTrans}):
   \begin{equation}
     (\boldsymbol{\underline{C}}_{Y,lin})_{N\times N} =
     \boldsymbol{\hat{Y}} \cdot
     \boldsymbol{\underline{C}}_{Z,lin} \cdot
     \boldsymbol{\hat{Y}}^+
   \end{equation}
   where $\boldsymbol{\hat{Y}}$ is taken from equation \ref{eqn:HBeqn}.
\end{enumerate}

Remark: If no explicit noise sources exist in the linear subcircuit,
$(\boldsymbol{\underline{C}}_{Z,lin})_{N\times N}$ can be computed
much faster by using Bosma's theorem (equation \ref{eqn:bosma}).


\section{The Non-Linear Subcircuit}

The noise in the non-linear part of the circuit is calculated
by using the quasi-static approach, i.e. for every moment in
time the voltages and currents are regarded as a time-dependend
bias point. The noise properties of these bias points are
used for the noise calculation.

\addvspace{12pt}

Remark: It is not clear whether this approach creates a valid result
for noise with long-time correlation (e.g. 1/f noise), too. But up
to now, no other methods were proposed and some publications
reported to have achieved reasonable results with this approach
and 1/f noise.

\addvspace{12pt}

