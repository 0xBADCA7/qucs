%
% This document contains the chapter about AC noise analysis.
%
% Copyright (C) 2003, 2004 Stefan Jahn <stefan@lkcc.org>
% Copyright (C) 2004 Michael Margraf <Michael.Margraf@alumni.TU-Berlin.DE>
%
% Permission is granted to copy, distribute and/or modify this document
% under the terms of the GNU Free Documentation License, Version 1.1
% or any later version published by the Free Software Foundation.
%

\chapter{AC Noise Analysis}

\section{The Concept}

To calculate the small signal noise of a circuit, the AC noise analysis
has to be applied \cite{Blum}. This technique uses the principle of
the AC analysis
described in chapter \ref{sec:acMNA} on page \pageref{sec:acMNA}. In
addition to the MNA matrix $A$ one needs the noise current correlation
matrix $\underline{C}_Y$ of the circuit, that contains the equivalent
noise current sources for every node on its main diagonal and their
correlation on the other positions.\\
The basic concept of the ac noise analysis is as follows: We want the
noise voltage at node $i$, so we calculate what voltage arises due to
the noise source at node $j$. We do that for every $n$ nodes and after
that we add all the noise voltages (by paying attention to their
correlation). But that would mean to solve the MNA equation $n$ times.
Fortunately there is a more easy way. One can perform the
above-mentioned $n$ steps in one single step, if the reciprocal MNA
matrix is used. This matrix equals the MNA matrix itself, if the
network is reciprocal. A network that only contains resistors,
capacitors, inductors and transformers is reciprocal.\\
The question that needs to be answered now is: How do we get the
reciprocal MNA matrix for an arbitrary network ? This is
equivalent to the question: How do we get the MNA matrix of the
adjoint network. The answer is quite simple: Just transpose the MNA
matrix!\\

For any network, calculating the noise voltage at node $i$
is done by the following three steps:
\begin{alignat}{2}
 & \text{1. Solving MNA equation:} & \qquad & \left[A\right]^T \cdot \left[x\right] =
\left[A\right]^T \cdot
\begin{bmatrix}
v \\
j \\
\end{bmatrix}
=
\begin{bmatrix}
0 \\
\vdots \\
0 \\
-1 \\
0 \\
\vdots \\
0 \\
\end{bmatrix}
\leftarrow i\text{-th row} \\
 & \text{2. Creating noise correlation matrix:} & \qquad
 & \left( \underline{C}_Y \right) \\
 & \text{3. Calculating noise voltage:} & \qquad
 & u_{noise,i} = \sqrt{\left[v\right]^T \cdot \left( \underline{C}_Y \right) \cdot \left[v\right]^*}
\end{alignat}
If the normal AC analysis has already be done with LU decomposition,
then the most time consuming work of step 1 has already be done.
\begin{equation}
\text{instead of} \qquad Y = L\cdot U \qquad \text{we have} \qquad
Y^T = U^T \cdot L^T
\end{equation}
I.e. $U^T$ becomes the new $L$ matrix and $L^T$ becomes the new $U$
matrix, and we do not need to solve the matrix equation again,
because only the right-hand side was changed. So altogether this
is a quickly done task. (Note that in step 3, only the subvector
$[v]$ of vector $[x]$ is used. See section \ref{sec:xmatrix} for
details on this.)\\
If we want to know the noise voltage at another node, only the
right-hand side of step 1 changes. That is, a new LU decomposition
is not needed.\\
As the noise current correlation matrix contains lots of zeros, it
is not worthy to create it in the computer memory. Instead, during
its creation, one can directly apply it to the $\left[v\right]^T$
vector.


\section{Noise Current Correlation Matrix}

This section describes the noise current correlation matrixes of
noisy components. The equations are built for RMS noise currents
with 1Hz bandwidth.\\ \\

Resistor with resistance $R$ and temperature $T$:
\begin{equation}
(\underline{C}_Y) = \frac{4\cdot k\cdot T}{R} \cdot
\begin{pmatrix}
 1 & -1 \\
-1 &  1 \\
\end{pmatrix}
\end{equation}

Noise current source with a current power spectral density of
$cPSD$:
\begin{equation}
(\underline{C}_Y) = cPSD \cdot
\begin{pmatrix}
 1 & -1 \\
-1 &  1 \\
\end{pmatrix}
\end{equation}

A noise voltage source (voltage power spectral density $vPSD$)
cannot be modeled with the noise current
matrix. That is why one has to use a noise current source
(current power spectral density $cPSD$) connected to a current
controlled voltage source (transimpedance $R$) satisfying the
equation
\begin{equation}
vPSD = cPSD \cdot R
\end{equation}

Attenuator with (power) attenuation $L$, reference impedance
$Z_{ref}$ and temperature $T$:
\begin{equation}
(\underline{C}_Y) = 4\cdot k\cdot T\cdot \text{Re}(\underline{Y})
 = \frac{4\cdot k\cdot T}{Z_{ref}\cdot (L-1)} \cdot
\begin{pmatrix}
 L+1            & -2\cdot\sqrt{L} \\
-2\cdot\sqrt{L} &  L+1 \\
\end{pmatrix}
\end{equation}

Isolator with reference impedance $Z_1$ (input) and $Z_2$ (output)
and temperature $T$:
\begin{equation}
(\underline{C}_Y) = 4\cdot k\cdot T\cdot
\begin{pmatrix}
 1/Z_1                 & 0 \\
-2/\sqrt{Z_1\cdot Z_2} &  1/Z_2 \\
\end{pmatrix}
\end{equation}

Diode (for details on the parameters see section \ref{sec:nw_diode}):
\begin{equation}
(\underline{C}_Y)
 = 2\cdot e\cdot K\cdot \left(I_{d} + 2\cdot I_{S}\right)\cdot
\begin{pmatrix}
   1 & -1\\
  -1 &  1\\
\end{pmatrix}\\
\end{equation}
